% enable page numbering
\pagenumbering{arabic}
%start counting here 
\setcounter{page}{1}
% header for pages
\pagestyle{fancy}
\fancyhead[L]{\myassignment  \hspace*{1.5mm}- \mycourse} 
\fancyhead[R]{\mydate}


%%% STARTE HERE
\section{Question 1 (10 Marks)}
Given a query that a user submits to an IR system and the top N documents
that are returned as relevant by the system, devise an approach (high-level
algorithmic steps will suffice)

to suggest query terms to add to the query. Typically, we wish to give a
large range of suggestions to the users capturing potential intended query
needs, i.e., high diversity of terms that may capture the intended query
context/content.


Ziel: query expanden können - which makes the querey better 
the terms should be relevant (able to expand the query in a meaningful way - find the correct topic), diverse (cover different topics)

we should design a heuristic allowing that

Ansatz: 
clusters, there are papers on that 



1. query with q\_n terms, for the whole vocuabulary we get the most similar terms (maybe 1000 t\_n)
2. cluster the t\_n terms into k clusters - then we get the most central term from every cluster

no terms that were in the query 

how to get the most describtive terms for the query 


when - runtime vs offline


there are multiple expansion methods out here: 
like sysnonm,  related term, contextual, 

recall vs precision tradeoff what we trying to solve and how we are acutal doing it

depending what a system we are designing (incooperate user data, e.g. Galway, Ireland, Europe)

user can give a temperature - to select the cluster

\section{Section TBD.}
\lipsum[1-4]


